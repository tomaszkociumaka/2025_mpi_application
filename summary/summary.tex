
\documentclass[a4paper,11pt]{article}
\usepackage{fancyhdr}
\usepackage[margin=2.5cm,nofoot]{geometry}
\usepackage[utf8]{inputenc}
\usepackage{hyperref}
\usepackage[OT4]{fontenc}
\usepackage{amsmath,amssymb}
\usepackage[dvipsnames]{xcolor}
\usepackage{setspace}
\usepackage{fontspec}
\setmainfont{Times New Roman}
\doublespacing
\hypersetup{
    colorlinks = true,
    citecolor = OliveGreen
}
\hypersetup{pdftitle={Tomasz Kociumaka: Research Summary}}

\usepackage{titlesec}

\titleformat{\section}  % which section command to format
  {\fontsize{14}{16}\bfseries} % format for whole line
  {\thesection} % how to show number
  {1em} % space between number and text
  {} % formatting for just the text
  [] % formatting for after the text

\newcommand{\eps}{\varepsilon}

\pagestyle{fancy}
\renewcommand{\headrulewidth}{0pt}
\lhead{Tomasz Kociumaka}
\rhead{Research Summary}

\fancypagestyle{firststyle}
{
  \renewcommand{\headrulewidth}{0pt}
  \lhead{Tomasz Kociumaka}
  \rhead{\today}
}

\newcommand{\Oh}{\mathcal{O}}
\newcommand{\Ohtilde}{\tilde{\mathcal{O}}}


\begin{document}
\thispagestyle{firststyle}
\begin{center}
{\bfseries {Research Summary}}
\end{center}\vspace{-.25cm}

Strings—sequences of characters over an alphabet—are among the most fundamental data types. They underlie tasks as diverse as web search, genome analysis, and file storage. The overarching goal of my research program is to design efficient algorithms for processing such strings, with a particular focus on fundamental theoretical problems with strong practical motivation.

A central example is approximate pattern matching. This problem models the task of finding a word in a document, but with tolerance for errors: we want to locate all fragments of text that resemble a given pattern, even if a few characters differ due to typos or mutations. Allowing such errors raises a fundamental algorithmic question: does the task become much harder? Classic results showed that when only a very small number of errors are allowed, one can still solve the problem as quickly as exact matching. For larger error rates, however, strong evidence suggests that no such fast solution exists. The precise dividing line between these two regimes has remained elusive. My research has pushed this boundary significantly forward, showing that efficient algorithms extend much further than previously known. Ongoing work continues to close the gap toward the conjectured limits, with the aim of finally settling where the true threshold lies.

Another key direction concerns lossless data compression. Compression is not only essential for storage and transmission, but increasingly for direct computation: large datasets should ideally remain compressed even while being queried. A prime example is human genomes, which are individually large but highly repetitive across a population. It is possible, in principle, to store such collections in a compressed form while still supporting queries about the sequences, but existing methods are either too slow or too complex, especially when datasets evolve over time (e.g., when new genomes are added). My work seeks both to develop simpler and more practical techniques for such scenarios, and to understand the fundamental trade-offs between how compactly we can store data and how efficiently we can access it.

The Max Planck Institute for Informatics is the natural host for this program. It is a world leader in algorithms and complexity, with deep expertise in areas closely connected to my work, including formal languages, data structures, and bioinformatics. The collaborative environment of the Saarland Informatics Campus offers exceptional opportunities to attract talent and foster interdisciplinary exchange. With Prof. Bringmann’s upcoming move to ETH Zurich, my group would strengthen and expand MPI’s leading role in the fine-grained complexity of string problems. Support for a Max Planck Research Group would allow me to build a cohesive team—recruiting doctoral students and postdoctoral researchers—and to pursue projects at a scale not possible in my current position, complementing the strong network of external collaborators on which I presently rely.

\end{document}
 
