
\documentclass[a4paper,11pt]{article}
\usepackage{fancyhdr}
\usepackage[margin=2.5cm]{geometry}
\usepackage[utf8]{inputenc}
\usepackage{hyperref}
\usepackage[OT4]{fontenc}
\usepackage{amsmath,amssymb}
\usepackage[dvipsnames]{xcolor}
\usepackage{setspace}
\usepackage{fontspec}
\setmainfont{Times New Roman}
\doublespacing
\hypersetup{
    colorlinks = true,
    citecolor = OliveGreen
}
\hypersetup{pdftitle={Tomasz Kociumaka: Research Summary}}

\usepackage{titlesec}

\titleformat{\section}  % which section command to format
  {\fontsize{14}{16}\bfseries} % format for whole line
  {\thesection} % how to show number
  {1em} % space between number and text
  {} % formatting for just the text
  [] % formatting for after the text

\newcommand{\eps}{\varepsilon}

\pagestyle{fancy}
\renewcommand{\headrulewidth}{0pt}
\lhead{Tomasz Kociumaka}
\rhead{Research Summary}

\fancypagestyle{firststyle}
{
  \renewcommand{\headrulewidth}{0pt}
  \lhead{Tomasz Kociumaka}
  \rhead{\today}
    \chead{\textbf{Research Summary}}
}

\newcommand{\Oh}{\mathcal{O}}
\newcommand{\Ohtilde}{\tilde{\mathcal{O}}}


\begin{document}
\thispagestyle{firststyle}

Strings—sequences of characters over an alphabet—are among the most fundamental data types. They underlie diverse tasks such as web search, genome analysis, and file storage. The overarching goal of my research program is to design efficient algorithms and data structures for processing strings, with a particular focus on fundamental theoretical problems with strong practical motivation.

An illustrative example is approximate pattern matching, which models the task of finding a word in a document, but with tolerance for errors: we want to locate all fragments of text that resemble a given pattern, even if some characters are inserted, deleted, or changed (e.g., due to typos or mutations). A fundamental algorithmic question is whether allowing errors makes the task much harder. Classic results from the 1990s show that when only a few errors are allowed, one can still solve the problem in time proportional to the input size, that is, essentially as quickly as exact pattern matching. For significantly larger error rates, however, strong evidence suggests that no such fast solution exists. My recent work provides efficient algorithms that support much more errors than previously known alternatives, but the precise dividing line between these two regimes remains elusive. One of the goals of my proposal is to determine this threshold and understand how the complexity degrades as the number of errors grows.

Another key direction concerns lossless data compression, which is essential not only for storage and transmission but increasingly also for computation: large repetitive datasets should ideally remain compressed even while being processed. Prime examples are collections of human genomes, which are highly similar across a population. It is possible, in principle, to store such datasets in a compressed form while still supporting queries about the underlying strings, and my previous work contributed representations with the smallest known overhead compared to widely used formats such as \texttt{gzip}. Still, existing methods remain very complex and relatively slow, especially when datasets change over time (e.g., when new genomes are added). My future work will focus on developing simpler and more practical techniques, as well as clarifying the fundamental theoretical trade-offs between storage size and query efficiency.

The Max Planck Institute for Informatics is the natural host for this program. With a world-leading department in algorithms and complexity, the broader area encompassing my research, it offers an exceptional environment for advancing fundamental work and attracting outstanding junior scientists. My agenda is distinct from those of other group leaders and faculty within the institute and the Saarland Informatics Campus, yet it benefits from proximity to experts in adjacent areas such as data structures, formal languages, and algorithmic bioinformatics. Establishing a Max Planck Research Group would transform my departmental role into a visible, independent program and position MPI as a hub for cutting-edge research in stringology. With a dedicated team of doctoral students and postdoctoral researchers, complementing my strong international network of collaborators, \mbox{I could pursue projects not feasible otherwise}.
\end{document}